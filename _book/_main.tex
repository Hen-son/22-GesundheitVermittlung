% Options for packages loaded elsewhere
\PassOptionsToPackage{unicode}{hyperref}
\PassOptionsToPackage{hyphens}{url}
\PassOptionsToPackage{dvipsnames,svgnames,x11names}{xcolor}
%
\documentclass[
  ngerman,
]{article}
\title{Studienprojekt Gesundheitsgerechtigkeit - Kritik und Vermittlung}
\usepackage{etoolbox}
\makeatletter
\providecommand{\subtitle}[1]{% add subtitle to \maketitle
  \apptocmd{\@title}{\par {\large #1 \par}}{}{}
}
\makeatother
\subtitle{Institut für Sozialwissenschaften \& Geographisches Institut, Humboldt-Universität zu Berlin, SoSe 2022}
\author{PD Dr.~Henning Füller (\href{mailto:henning.fueller@geo.hu-berlin.de}{\nolinkurl{henning.fueller@geo.hu-berlin.de}}) \and PD Dr.~Henrik Lebuhn (\href{mailto:henrik.lebuhn@sowi.hu-berlin.de}{\nolinkurl{henrik.lebuhn@sowi.hu-berlin.de}})}
\date{wöchentlich, Montag 10:00 Uhr - 14:00 Uhr, Institut für Sozialwissenschaften, Universitätsstrasse 3b, Raum 001}

\usepackage{amsmath,amssymb}
\usepackage{lmodern}
\usepackage{iftex}
\ifPDFTeX
  \usepackage[T1]{fontenc}
  \usepackage[utf8]{inputenc}
  \usepackage{textcomp} % provide euro and other symbols
\else % if luatex or xetex
  \usepackage{unicode-math}
  \defaultfontfeatures{Scale=MatchLowercase}
  \defaultfontfeatures[\rmfamily]{Ligatures=TeX,Scale=1}
\fi
% Use upquote if available, for straight quotes in verbatim environments
\IfFileExists{upquote.sty}{\usepackage{upquote}}{}
\IfFileExists{microtype.sty}{% use microtype if available
  \usepackage[]{microtype}
  \UseMicrotypeSet[protrusion]{basicmath} % disable protrusion for tt fonts
}{}
\makeatletter
\@ifundefined{KOMAClassName}{% if non-KOMA class
  \IfFileExists{parskip.sty}{%
    \usepackage{parskip}
  }{% else
    \setlength{\parindent}{0pt}
    \setlength{\parskip}{6pt plus 2pt minus 1pt}}
}{% if KOMA class
  \KOMAoptions{parskip=half}}
\makeatother
\usepackage{xcolor}
\IfFileExists{xurl.sty}{\usepackage{xurl}}{} % add URL line breaks if available
\IfFileExists{bookmark.sty}{\usepackage{bookmark}}{\usepackage{hyperref}}
\hypersetup{
  pdftitle={Studienprojekt Gesundheitsgerechtigkeit - Kritik und Vermittlung},
  pdfauthor={PD Dr.~Henning Füller (henning.fueller@geo.hu-berlin.de); PD Dr.~Henrik Lebuhn (henrik.lebuhn@sowi.hu-berlin.de)},
  pdflang={de},
  colorlinks=true,
  linkcolor={Maroon},
  filecolor={Maroon},
  citecolor={Blue},
  urlcolor={blue},
  pdfcreator={LaTeX via pandoc}}
\urlstyle{same} % disable monospaced font for URLs
\usepackage[margin=1in]{geometry}
\usepackage{longtable,booktabs,array}
\usepackage{calc} % for calculating minipage widths
% Correct order of tables after \paragraph or \subparagraph
\usepackage{etoolbox}
\makeatletter
\patchcmd\longtable{\par}{\if@noskipsec\mbox{}\fi\par}{}{}
\makeatother
% Allow footnotes in longtable head/foot
\IfFileExists{footnotehyper.sty}{\usepackage{footnotehyper}}{\usepackage{footnote}}
\makesavenoteenv{longtable}
\usepackage{graphicx}
\makeatletter
\def\maxwidth{\ifdim\Gin@nat@width>\linewidth\linewidth\else\Gin@nat@width\fi}
\def\maxheight{\ifdim\Gin@nat@height>\textheight\textheight\else\Gin@nat@height\fi}
\makeatother
% Scale images if necessary, so that they will not overflow the page
% margins by default, and it is still possible to overwrite the defaults
% using explicit options in \includegraphics[width, height, ...]{}
\setkeys{Gin}{width=\maxwidth,height=\maxheight,keepaspectratio}
% Set default figure placement to htbp
\makeatletter
\def\fps@figure{htbp}
\makeatother
\setlength{\emergencystretch}{3em} % prevent overfull lines
\providecommand{\tightlist}{%
  \setlength{\itemsep}{0pt}\setlength{\parskip}{0pt}}
\setcounter{secnumdepth}{5}
\newlength{\cslhangindent}
\setlength{\cslhangindent}{1.5em}
\newlength{\csllabelwidth}
\setlength{\csllabelwidth}{3em}
\newlength{\cslentryspacingunit} % times entry-spacing
\setlength{\cslentryspacingunit}{\parskip}
\newenvironment{CSLReferences}[2] % #1 hanging-ident, #2 entry spacing
 {% don't indent paragraphs
  \setlength{\parindent}{0pt}
  % turn on hanging indent if param 1 is 1
  \ifodd #1
  \let\oldpar\par
  \def\par{\hangindent=\cslhangindent\oldpar}
  \fi
  % set entry spacing
  \setlength{\parskip}{#2\cslentryspacingunit}
 }%
 {}
\usepackage{calc}
\newcommand{\CSLBlock}[1]{#1\hfill\break}
\newcommand{\CSLLeftMargin}[1]{\parbox[t]{\csllabelwidth}{#1}}
\newcommand{\CSLRightInline}[1]{\parbox[t]{\linewidth - \csllabelwidth}{#1}\break}
\newcommand{\CSLIndent}[1]{\hspace{\cslhangindent}#1}
\AtBeginDocument{\renewcommand{\contentsname}{Seminarplan}} \usepackage{titling} \pretitle{\begin{center} \hfill\includegraphics[width=1.5in]{husiegel.pdf}\LARGE\\\vspace{0.8cm}} \posttitle{\end{center}}
\ifXeTeX
  % Load polyglossia as late as possible: uses bidi with RTL langages (e.g. Hebrew, Arabic)
  \usepackage{polyglossia}
  \setmainlanguage[]{german}
\else
  \usepackage[main=ngerman]{babel}
% get rid of language-specific shorthands (see #6817):
\let\LanguageShortHands\languageshorthands
\def\languageshorthands#1{}
\fi
\ifLuaTeX
  \usepackage{selnolig}  % disable illegal ligatures
\fi

\begin{document}
\maketitle

{
\hypersetup{linkcolor=}
\setcounter{tocdepth}{2}
\tableofcontents
}
\newpage

\hypertarget{allgemeines}{%
\section*{Allgemeines}\label{allgemeines}}
\addcontentsline{toc}{section}{Allgemeines}

\begin{itemize}
\item
  Seminarorganisation erfolgt über Moodle - \href{https://moodle.hu-berlin.de/course/view.php?id=110268}{Gesundheitsgerechtigkeit - Kritik und Vermittlung} (Einschreibeschlüssel: Virchow).
\item
  Das Seminar findet wöchentlich Montags von 10:15 Uhr bis 13:45 Uhr statt. Raum:
\item
  Als unterstützendes Whiteboard für die Visualisierung der Diskussion nutzen wir das \href{https://miro.com/app/board/uXjVOF6MyAc=/?invite_link_id=891179698806}{Miroboard - SP Gesundheitsgerechtigkeit}.
\end{itemize}

\hypertarget{zielsetzung}{%
\section*{Zielsetzung}\label{zielsetzung}}

Covid hat erneut ein Schlaglicht auf die Bedeutung von räumlichen Faktoren für Fragen von Gesundheit und sozialer Gerechtigkeit geworfen. In Stadtplanung und Public Health wird neuerdings verstärkt ein Setting-Ansatz propagiert, der die Bedeutung von Nachbarschaft, räumlichem Umfeld und Umweltfaktoren für Gesundheit und Wohlergehen unterstreicht. Berlin hat bereits 2016 Umweltgerechtigkeit zu einer Zielvorgabe gemacht und nimmt hier bundes- und europaweit eine Vorreiterrolle ein. Auch die feministischen Debatten um Care und Sorgearbeit werden in den vergangenen Jahren verstärkt in der Stadtforschung, Raumplanung und Architektur rezipiert. Zugleich zeigt sich, dass die Messung, Kategorisierung und sozialräumliche Verortung von Strukturen, Bedingungen und Prozessen auf dem Feld der Gesundheitsforschung mit großen Herausforderungen verbunden ist.

In dem Seminar wollen wir uns zunächst in die aktuelle Literatur einlesen (Environmental Justice, Kritische Sozialepidemiologie, Kritische Kartographie, stadtsoziologische Care-Ansätze) und mit den konzeptionellen und empirischen Schwierigkeiten auseinandersetzen. Auf dieser Grundlage wollen wir dann eigene mediale Formate entwickeln und erstellen, die das Thema Umweltgerechtigkeit für eine breitere Öffentlichkeit an greifbaren Beispielen zugänglich und verständlich machen, und dabei auch die impliziten Voraussetzungen kritisch reflektieren. Denkbar wäre unter anderem die Konzeption einer Wanderausstellung in den Berliner Stadtteilbibliotheken oder auch die Zusammenstellung einer Handreichung für Lehrkräfte oder politische Bildner*innen.

Das Seminar wird im co-teaching Format unterrichtet und wendet sich vor allem an Studierende der Geographie und der Sozialwissenschaften.

\hypertarget{literatur-zum-einstieg}{%
\subsection*{Literatur zum Einstieg}\label{literatur-zum-einstieg}}

\begin{itemize}
\tightlist
\item
  Dzudzek, Iris und Anke Strüver. 2020. Urbane Gesundheitsgerechtigkeit. \emph{Geographische Zeitschrift}. doi: \href{https://doi.org/10.25162/gz-2020-0005}{10.25162/gz-2020-0005} .
\item
  Gabauer, Angelika et al (2022): \emph{Care and the City. Encounters with Urban Studies}, Routledge.
\item
  Orangotango. 2018. \emph{This is not an Atlas. A Global Collection of Counter-Cartographies}. Bielefeld: transcript.
\end{itemize}

\newpage

\hypertarget{anforderungen}{%
\section*{Anforderungen}\label{anforderungen}}

Bei einem Studienprojekt steht die projekthafte Erarbeitung im Vordergrund. In dem Kurs geht es uns einerseits darum, den Zusammenhang von sozialen Strukturen, Gesundheit und Raum besser verstehen. Andererseits wollen wir uns kritisch mit der Stigmatisierung durch Karten und Verräumlichungen auseinander setzen. Die Beschäftigung mündet in die Erarbeitung eines Materialpakets für den Schulunterricht in der Sekundarstufe. Modulabschlussprüfung ist ein kommentiertes Paket mit Unterrichtsmaterialien (Gruppenarbeit) das über den Bildungsserver Berlin Brandenburg (Kooperationsvertrag) online für interessierte Lehrkräfte zur Verfügung gestellt wird.

\hypertarget{uxfcberblick-leistungsanforderungen}{%
\subsection*{Überblick Leistungsanforderungen}\label{uxfcberblick-leistungsanforderungen}}

\begin{enumerate}
\def\labelenumi{\arabic{enumi}.}
\item
  \textbf{Pflichtlektüre} zu jeder Sitzung ist verpflichtend von allen zu lesen als Basis für die Diskussion.
\item
  \textbf{schriftlicher Lektürekommentar} oder \textbf{Kurzreferat} Jede/r Teilnehmer*in schreibt im Laufe des Seminars drei kurze Lektürekommentare (ca. 1 Seite) zu einer der angegebenen Pflichtlektüren oder bereitet zu einer Sitzung einen kurzen mündlichen Input zu einem der ergänzenden Texte vor. Keine Textzusammenfassung, sondern eigene Fragen bzw. Kritik/Kommentare als Anstoß für die Diskussion. Die schriftlichen Kommentare/Präsentationen des mündlichen Inputs sollen am Tag vor der betreffenden Sitzung selbständig auf Moodle bereit gestellt werden.
\item
  \textbf{Modulabschlussprüfung} Das Modul wird mit einer Projektarbeit abgeschlossen (Gruppenarbeit). Ziel ist Lehrmaterial zu erarbeiten, dass auf dem Berliner Bildungsserver für Lehrkräfte in Sekundarschulen zur Unterrichtsvorbereitung genutzt werden kann.
\end{enumerate}

\begin{itemize}
\tightlist
\item
  Thema und Umfang wird gemeinsam im Seminar konkretisiert.
\item
  Abgabe als pdf bis zum 30.09.2022 via Moodle
\end{itemize}

\pagebreak

\hypertarget{seminarsitzungen}{%
\section*{Seminarsitzungen}\label{seminarsitzungen}}
\addcontentsline{toc}{section}{Seminarsitzungen}

\hypertarget{sitzung-am-25.04.-einfuxfchrung}{%
\subsection*{1. Sitzung am 25.04. -- Einführung}\label{sitzung-am-25.04.-einfuxfchrung}}
\addcontentsline{toc}{subsection}{1. Sitzung am 25.04. -- Einführung}

In der ersten Sitzung besprechen wir Seminarplan, Seminarorganisation und Leistungsanforderungen. Wir verabreden die Termine für Präsentationen der Teilnehmer:innen.

\hypertarget{textarbeit}{%
\subsubsection*{Textarbeit}\label{textarbeit}}
\addcontentsline{toc}{subsubsection}{Textarbeit}

\textbf{Lesetexte für alle:}

\begin{itemize}
\item
  Einführung in das Konzept der Sozialepidemiologie; Unterscheidung Pathogenese/Salutogenese:

  Schott, Thomas und Benjamin Kuntz. (2011). Sozialepidemiologie: Über die Wechselwirkungen von Gesundheit und Gesellschaft. In: \emph{Die Gesellschaft und ihre Gesundheit}, 159--171. Wiesbaden: VS Verlag für Sozialwissenschaften. \url{doi:\%5B10.1007/978-3-531-92790-9_8}{]}(\url{https://doi.org/10.1007/978-3-531-92790-9_8}).
\end{itemize}

\hypertarget{konzepte}{%
\section*{Konzepte}\label{konzepte}}
\addcontentsline{toc}{section}{Konzepte}

\hypertarget{sitzung-am-02.05.-gesundheit-und-ungleichheit}{%
\subsection*{2. Sitzung am 02.05. \textbar{} Gesundheit und Ungleichheit}\label{sitzung-am-02.05.-gesundheit-und-ungleichheit}}
\addcontentsline{toc}{subsection}{2. Sitzung am 02.05. \textbar{} Gesundheit und Ungleichheit}

In der ersten inhaltlichen Sitzung verschaffen wir uns einen grundlegenden Einblick in Perspektive der Sozialepidemiologie. Gesundheit ist hier als ein kollektives Phänomen interessant und im Vordergrund stehen die gesellschaftlichen Ursachen, die Verteilung und die Folgen von Gesundheit und Krankheit.

\hypertarget{textarbeit-1}{%
\subsubsection*{Textarbeit}\label{textarbeit-1}}
\addcontentsline{toc}{subsubsection}{Textarbeit}

\textbf{Lesetext für alle:}

\begin{itemize}
\item
  Gesundheit als eine zentrale Dimension gesellschaftlicher Ungleichheit. Vorschlag einer kritischen sozialepidemiologischen Forschungsagenda:

  Krieger, Nancy. (2019). A Critical Research Agenda for Social Justice and Public Health: An Ecosocial Proposal. 531-\/-552. In: \emph{Social Injustice and Public Health}, hg. von Barry S. Levy und Heather L. McStowe. Oxford; New York: Oxford University Press.
\end{itemize}

\textbf{weitere Lesetexte}
(mindestens einer dieser Texte zusätzlich verpflichtend von Allen zu lesen)

\begin{itemize}
\item
  Vorschlag einer relationalen Perspektive auf Körper/Gesellschaft/Umwelt (eher konzeptionell ausgerichtet, Hintergrund):

  Pálsson, Gísli. (2013). Ensembles of biosocial relations. In: \emph{Biosocial becomings: integrating social and biological anthropology}, hg. von Tim Ingold und Pálsson Gísli. Cambridge und New York: Cambridge University Press.
\item
  Urbane Gesundheit relational denken (eher beispielhaft ausgerichtet, Anwendung):

  Dzudzek, Iris und Anke Strüver. (2020). Urbane Gesundheitsgerechtigkeit. \emph{Geographische Zeitschrift} 108, Nr. 4: 249--271. \url{doi:10.25162/gz-2020-0005}
\end{itemize}

\hypertarget{projektarbeit}{%
\subsubsection*{Projektarbeit}\label{projektarbeit}}
\addcontentsline{toc}{subsubsection}{Projektarbeit}

\begin{itemize}
\tightlist
\item
  Kennenlernen über Stadtplan-Spiel
\item
  Diskussion entlang aktueller Zeitungsartikel
\end{itemize}

\hypertarget{sitzung-am-09.05.-umweltgerechtigkeit-in-der-stadt}{%
\subsection*{3. Sitzung am 09.05. \textbar{} Umweltgerechtigkeit in der Stadt}\label{sitzung-am-09.05.-umweltgerechtigkeit-in-der-stadt}}
\addcontentsline{toc}{subsection}{3. Sitzung am 09.05. \textbar{} Umweltgerechtigkeit in der Stadt}

In dieser Sitzung wollen wir uns kritisch mit dem Ansatz der gesundheitsfördernden Raumplanung auseinander setzen. Wie lässt sich Gesundheit in der Stadt messen und planerisch bearbeiten? Was sind Fallstricke dabei?

\hypertarget{textarbeit-2}{%
\subsubsection*{Textarbeit}\label{textarbeit-2}}
\addcontentsline{toc}{subsubsection}{Textarbeit}

\textbf{Pflichtlektüre}

\begin{itemize}
\item
  Kritische Auseinandersetzung mit der aktuellen Bearbeitung von Gesundheit in der Stadtplanung. Vorstellung des SUHEI (Spatial Urban Health Equity Indicators) Modells -- derzeit ein dominantes Screening-Werkzeug um das Leitbild einer gesundheitsfördernden räumlichen Planung umzusetzen:

  Köckler, Heike, Daniel Simon, Kerstin Agatz und Johannes Flacke. (2020). Gesundheitsfördernde Stadtentwicklung. \emph{Informationen zur Raumentwicklung} 47, Nr. 1: 96--109.
\end{itemize}

\textbf{weitere Lesetexte}
(mindestens einer dieser Texte zusätzlich verpflichtend von Allen zu lesen)

\begin{itemize}
\item
  Plädoyer für eine städtebauliche Aufwertungsstrategie, die Gesundheitsförderung und soziale Gerechtigkeit gleichermaßen im Auge behält:

  Wolch, Jennifer R., Jason Byrne und Joshua P. Newell. (2014). Urban green space, public health, and environmental justice: The challenge of making cities ‚just green enough'. \emph{Landscape and Urban Planning} 125. \url{doi:10.1016/j.landurbplan.2014.01.017}.
\item
  Demenzerkrankung und besondere Bedarfe in der Stadtplanung:

  Biglieri, Samantha. (2021). Examining Everyday Outdoor Practices in Suburban Public Space: The Case for an Expanded Definition of Care as an Analytical Framework. In: \emph{Care and the City. Encounters with Urban Studies}. London: Routldege.
\end{itemize}

\hypertarget{projektarbeit-1}{%
\subsubsection*{Projektarbeit}\label{projektarbeit-1}}
\addcontentsline{toc}{subsubsection}{Projektarbeit}

\begin{itemize}
\tightlist
\item
  Vorstellung Bildungsserver Berlin/Brandenburg
\item
  Gruppenfindung / Festlegung der Themen
\end{itemize}

\hypertarget{sitzung-am-16.05.-visualisierung-des-sozialen}{%
\subsection*{4. Sitzung am 16.05. \textbar{} Visualisierung des Sozialen}\label{sitzung-am-16.05.-visualisierung-des-sozialen}}
\addcontentsline{toc}{subsection}{4. Sitzung am 16.05. \textbar{} Visualisierung des Sozialen}

In dieser Sitzung schaffen wir die Grundlage für eine kritische Auseinandersetzung mit räumlichen Determinanten von Gesundheit in der Stadt. Wie lassen sich biosoziale Phänomene darstellen? Welchen Einfluss haben die Werkzeuge der Messung, Kategorisierung und Visualisierung?

\hypertarget{textarbeit-3}{%
\subsubsection*{Textarbeit}\label{textarbeit-3}}
\addcontentsline{toc}{subsubsection}{Textarbeit}

\textbf{Lesetext für alle:}

\begin{itemize}
\item
  Grundlegender Aufsatz zu dem ‚Doppelschlag` kritischer Kartographie -- Kritik der Macht von Karten und Entwicklung kritischer Karten:

  Crampton, Jeremy W. und John Krygier. (2005). An Introduction to Critical Cartography. \emph{ACME} 4, Nr. 1: 11--33.
\end{itemize}

\textbf{weitere Lesetexte}
(mindestens einer dieser Texte zusätzlich verpflichtend von Allen zu lesen)

\begin{itemize}
\item
  Konzept und Beispiele einer ‚Gegen-Kartographie`:

  Mesquita, André. (2018). Counter-Cartographies. Politics, Art and the Insurrection of Maps. In: \emph{This is not an Atlas. A Global Collection of Counter-Cartographies}, hg. von Orangotango, 26--30. Bielefeld: transcript.
\item
  Mit Karten werden Konflikte ausgetragen. Das Beispiel Tempelhofer Feld

  Bittner, Christian und Boris Michel. (2014). Kritische Kartographien der Stadt. In: \emph{Handbuch kritische Stadtgeographie}, hg. von Bernd Belina, Matthias Naumann, und Anke Strüver, 64--70. Westfälisches Dampfboot.
\end{itemize}

\hypertarget{projektarbeit-2}{%
\subsubsection*{Projektarbeit}\label{projektarbeit-2}}
\addcontentsline{toc}{subsubsection}{Projektarbeit}

\begin{itemize}
\tightlist
\item
  Diskussion von Karten-Beispielen (angefr. Exkurs Robert Viehf)
\item
  Planung der Projekte // angefr. Gespräch mit Hendrik Struck (Herrmann-Hesse-Gymnasium)
\end{itemize}

\hypertarget{kontext}{%
\section*{Kontext}\label{kontext}}
\addcontentsline{toc}{section}{Kontext}

\hypertarget{sitzung-am-23.05.-berlin-i-gesundheit-bottom-up---exkursion-gesundheitszentrum}{%
\subsection*{5. Sitzung am 23.05. \textbar{} Berlin I Gesundheit bottom-up - Exkursion Gesundheitszentrum}\label{sitzung-am-23.05.-berlin-i-gesundheit-bottom-up---exkursion-gesundheitszentrum}}
\addcontentsline{toc}{subsection}{5. Sitzung am 23.05. \textbar{} Berlin I Gesundheit bottom-up - Exkursion Gesundheitszentrum}

\hypertarget{textarbeit-4}{%
\subsubsection*{Textarbeit}\label{textarbeit-4}}
\addcontentsline{toc}{subsubsection}{Textarbeit}

\textbf{Lesetext für Alle}

\begin{itemize}
\item
  Plädoyer für eine Ausweitung kritischer Stadtforschung auf das Themenfeld Gesundheit. Gesundheit und Krankheit als verkörperte Formen gesellschaftlicher Macht- und Ungleichheitsverhältnisse begreifen:

  Bůžek, Richard, Iris Dzudzek, Susanne Hübl und Lisa Kamphaus. (2022). Wenn die Verhältnisse unter die Haut gehen. \emph{sub\textbackslash urban. zeitschrift für kritische stadtforschung} 10, Nr. 1/2. doi: 10.36900/suburban.v10i1/2.702.
\end{itemize}

\hypertarget{projektarbeit-3}{%
\subsubsection*{Projektarbeit}\label{projektarbeit-3}}
\addcontentsline{toc}{subsubsection}{Projektarbeit}

\begin{itemize}
\tightlist
\item
  Vor-Ort-Begehung (angefr. Angela)
\end{itemize}

\hypertarget{sitzung-am-30.05.-berlin-ii-gesundheit-top-down---atlas-umweltgerechtigkeit}{%
\subsection*{6. Sitzung am 30.05. \textbar{} Berlin II Gesundheit top-down - Atlas Umweltgerechtigkeit}\label{sitzung-am-30.05.-berlin-ii-gesundheit-top-down---atlas-umweltgerechtigkeit}}
\addcontentsline{toc}{subsection}{6. Sitzung am 30.05. \textbar{} Berlin II Gesundheit top-down - Atlas Umweltgerechtigkeit}

\hypertarget{textarbeit-5}{%
\subsubsection*{Textarbeit}\label{textarbeit-5}}
\addcontentsline{toc}{subsubsection}{Textarbeit}

\textbf{Fallbeispiel Umweltgerechtigkeit}

Debatte um Umweltgerechtigkeit am Beispiel Berlin

\begin{itemize}
\tightlist
\item
  Plattform Umweltgerechtigkeit in Berlin -- \url{https://www.umweltgerechtigkeit-berlin.de}
\item
  Böhme, Christa und Thomas Franke. (2021). Umweltgerechtigkeit und Städtebauförderung. \emph{vhw FWS} 2: 71--74.
\end{itemize}

\hypertarget{projektarbeit-4}{%
\subsubsection*{Projektarbeit}\label{projektarbeit-4}}
\addcontentsline{toc}{subsubsection}{Projektarbeit}

\begin{itemize}
\tightlist
\item
  Themen der Projektgruppen festlegen
\item
  Rahmen diskutieren
\end{itemize}

\hypertarget{sitzung-am-06.06-entfuxe4llt-pfingsten}{%
\subsection*{Sitzung am 06.06 entfällt (Pfingsten)}\label{sitzung-am-06.06-entfuxe4llt-pfingsten}}
\addcontentsline{toc}{subsection}{Sitzung am 06.06 entfällt (Pfingsten)}

\hypertarget{projekt}{%
\section*{Projekt}\label{projekt}}
\addcontentsline{toc}{section}{Projekt}

\hypertarget{sitzung-am-13.06.-eigenstuxe4ndige-projektarbeit}{%
\subsection*{7. Sitzung am 13.06. \textbar{} eigenständige Projektarbeit}\label{sitzung-am-13.06.-eigenstuxe4ndige-projektarbeit}}
\addcontentsline{toc}{subsection}{7. Sitzung am 13.06. \textbar{} eigenständige Projektarbeit}

-- \emph{Keine Präsenzsitzung} --

\hypertarget{projektarbeit-5}{%
\subsubsection*{Projektarbeit}\label{projektarbeit-5}}
\addcontentsline{toc}{subsubsection}{Projektarbeit}

\begin{itemize}
\tightlist
\item
  eigenständige Forschung / Recherche
\end{itemize}

\hypertarget{sitzung-am-20.06.-werkstattsitzung}{%
\subsection*{8. Sitzung am 20.06. \textbar{} Werkstattsitzung}\label{sitzung-am-20.06.-werkstattsitzung}}
\addcontentsline{toc}{subsection}{8. Sitzung am 20.06. \textbar{} Werkstattsitzung}

\hypertarget{projektwerkstatt}{%
\subsubsection*{Projektwerkstatt}\label{projektwerkstatt}}
\addcontentsline{toc}{subsubsection}{Projektwerkstatt}

\begin{itemize}
\tightlist
\item
  Vorstellung des aktuellen Stands in Einzelgesprächen der Gruppen mit den Dozenten
\end{itemize}

\hypertarget{sitzung-am-27.06.-eigenstuxe4ndige-projektarbeit}{%
\subsection*{9. Sitzung am 27.06. \textbar{} eigenständige Projektarbeit}\label{sitzung-am-27.06.-eigenstuxe4ndige-projektarbeit}}
\addcontentsline{toc}{subsection}{9. Sitzung am 27.06. \textbar{} eigenständige Projektarbeit}

-- \emph{Keine Präsenzsitzung} --

\hypertarget{projektarbeit-6}{%
\subsubsection*{Projektarbeit}\label{projektarbeit-6}}
\addcontentsline{toc}{subsubsection}{Projektarbeit}

\begin{itemize}
\tightlist
\item
  eigenständige Forschung / Recherche
\end{itemize}

\hypertarget{sitzung-am-04.07.-pruxe4sentation-gruppe-1-2-und-3}{%
\subsection*{10. Sitzung am 04.07. \textbar{} Präsentation Gruppe 1, 2 und 3}\label{sitzung-am-04.07.-pruxe4sentation-gruppe-1-2-und-3}}
\addcontentsline{toc}{subsection}{10. Sitzung am 04.07. \textbar{} Präsentation Gruppe 1, 2 und 3}

\hypertarget{projektarbeit-7}{%
\subsubsection*{Projektarbeit}\label{projektarbeit-7}}
\addcontentsline{toc}{subsubsection}{Projektarbeit}

\begin{itemize}
\tightlist
\item
  Präsentation Gruppe 1
\item
  Präsentation Gruppe 2
\item
  Präsentation Gruppe 3
\end{itemize}

\hypertarget{sitzung-am-11.07.-pruxe4sentation-gruppe-4-5-und-6}{%
\subsection*{11. Sitzung am 11.07. \textbar{} Präsentation Gruppe 4, 5 und 6}\label{sitzung-am-11.07.-pruxe4sentation-gruppe-4-5-und-6}}
\addcontentsline{toc}{subsection}{11. Sitzung am 11.07. \textbar{} Präsentation Gruppe 4, 5 und 6}

\hypertarget{projektarbeit-8}{%
\subsubsection*{Projektarbeit}\label{projektarbeit-8}}
\addcontentsline{toc}{subsubsection}{Projektarbeit}

\begin{itemize}
\tightlist
\item
  Präsentation Gruppe 4
\item
  Präsentation Gruppe 5
\item
  Präsentation Gruppe 6
\end{itemize}

\hypertarget{sitzung-am-18.07.-abschlusssitzung}{%
\subsection*{12. Sitzung am 18.07. \textbar{} Abschlusssitzung}\label{sitzung-am-18.07.-abschlusssitzung}}
\addcontentsline{toc}{subsection}{12. Sitzung am 18.07. \textbar{} Abschlusssitzung}

Zusammenfassung, Reflexion, Planung der Projektarbeiten

\hypertarget{literatur}{%
\section*{Literatur}\label{literatur}}
\addcontentsline{toc}{section}{Literatur}

\hypertarget{refs}{}
\begin{CSLReferences}{1}{0}
\leavevmode\vadjust pre{\hypertarget{ref-Biglieri21Examinin}{}}%
Biglieri, Samantha. 2021. {„Examining Everyday Outdoor Practices in Suburban Public Space: The Case for an Expanded Definition of Care as an Analytical Framework``}. In \emph{Care and the City. Encounters with Urban Studies}. London: Routldege.

\leavevmode\vadjust pre{\hypertarget{ref-Bittner14Kritische}{}}%
Bittner, Christian, und Boris Michel. 2014. {„Kritische Kartographien der Stadt``}. In \emph{Handbuch kritische Stadtgeographie}, herausgegeben von Bernd Belina, Matthias Naumann, und Anke Strüver, 64--70. {Westf{ä}lisches Dampfboot}.

\leavevmode\vadjust pre{\hypertarget{ref-Bohme21Umweltge}{}}%
Böhme, Christa, und Thomas Franke. 2021. {„Umweltgerechtigkeit und St{ä}dtebauf{ö}rderung``}. \emph{vhw FWS} 2: 71--74.

\leavevmode\vadjust pre{\hypertarget{ref-Buzek22Wenn-die}{}}%
Bůžek, Richard, Iris Dzudzek, Susanne Hübl, und Lisa Kamphaus. 2022. {„Wenn die Verh{ä}ltnisse unter die Haut gehen``}. \emph{sub{\textbackslash{}}urban. zeitschrift f{ü}r kritische stadtforschung} 10 (1/2). \url{https://doi.org/10.36900/suburban.v10i1/2.702}.

\leavevmode\vadjust pre{\hypertarget{ref-Crampton05An-Intro}{}}%
Crampton, Jeremy W., und John Krygier. 2005. {„An Introduction to Critical Cartography``}. \emph{ACME} 4 (1): 11--33.

\leavevmode\vadjust pre{\hypertarget{ref-Dzudzek20Urbane-G}{}}%
Dzudzek, Iris, und Anke Strüver. 2020. {„Urbane Gesundheitsgerechtigkeit``}. \emph{Geographische Zeitschrift} 108 (4): 249--71. \url{https://doi.org/10.25162/gz-2020-0005}.

\leavevmode\vadjust pre{\hypertarget{ref-Kockler20Gesundhe}{}}%
Köckler, Heike, Daniel Simon, Kerstin Agatz, und Johannes Flacke. 2020. {„Gesundheitsf{ö}rdernde Stadtentwicklung``}. \emph{Informationen zur Raumentwicklung} 47 (1): 96--109.

\leavevmode\vadjust pre{\hypertarget{ref-Krieger19A-Critic}{}}%
Krieger, Nancy. 2019. {„A Critical Research Agenda for Social Justice and Public Health: An Ecosocial Proposal``}. 531-\/-552. In \emph{Social Injustice and Public Health}, herausgegeben von Barry S. Levy und Heather L. McStowe. Oxford; New York: Oxford University Press.

\leavevmode\vadjust pre{\hypertarget{ref-Mesquita18Counter}{}}%
Mesquita, André. 2018. {„Counter-Cartographies. Politics, Art and the Insurrection of Maps``}. In \emph{This is not an Atlas. A Global Collection of Counter-Cartographies}, herausgegeben von Orangotango, 26--30. Bielefeld: transcript.

\leavevmode\vadjust pre{\hypertarget{ref-Palsson13Ensemble}{}}%
Pálsson, Gísli. 2013. {„Ensembles of biosocial relations``}. In \emph{Biosocial becomings: integrating social and biological anthropology}, herausgegeben von Tim Ingold und Pálsson Gísli. Cambridge und New York: {Cambridge University Press}.

\leavevmode\vadjust pre{\hypertarget{ref-Schott11Sozialep}{}}%
Schott, Thomas, und Benjamin Kuntz. 2011. {„Sozialepidemiologie: {Ü}ber die Wechselwirkungen von Gesundheit und Gesellschaft``}. In \emph{Die Gesellschaft und ihre Gesundheit}, 159--71. Wiesbaden: {VS} Verlag f{ü}r Sozialwissenschaften. \url{https://doi.org/10.1007/978-3-531-92790-9_8}.

\leavevmode\vadjust pre{\hypertarget{ref-Wolch14Urban}{}}%
Wolch, Jennifer R., Jason Byrne, und Joshua P. Newell. 2014. {„Urban green space, public health, and environmental justice: The challenge of making cities {‚just green enough`}``}. \emph{Landscape and Urban Planning} 125. \url{https://doi.org/10.1016/j.landurbplan.2014.01.017}.

\end{CSLReferences}

\end{document}
